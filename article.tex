
\documentclass[article,a4paper]{IEEEtran}
\usepackage{lipsum}
\usepackage[backend=biber]{biblatex}

\addbibresource{refs.bib}
\title{Internet of things areas of application}
\author{
\IEEEauthorblockN{Anton Odén}\\
\IEEEauthorblockA{Dept. of Maths and Computer Science\\Karlstad University\\
651 88 KARLSTAD, Sweden}\\
anton.oden@outlook.com
}

\begin{document}

\maketitle
\begin{abstract}
    \end{abstract}
    
    \section{Introduction}
    
    \section{Industry 4.0}
    Industry 4.0 is the reference to the fourth revolution of the industry that is hard to fully define as we are in the revolution today. The first revolution was when industry started to use machines powered by steam. Second revolution was the electrifing of industries. Electric power made industry location less dependent on position compared to first revolution that was dependent on the energy to be generated at the same location as industry. Electric power compared to steam could be transported. Third industry revolution is the digitalization of industry which we also could say that we still are in but the definition is clearer as the revolution has been going on since the around 1960. Digital record-keeping and global connectivity has accelerated the exchange of information, goods and services. Introduction of automated processes and robotics has made efficience improvements unto manufacturing and logistics. 
    The fourth industry revolution is the connectivity between all parts of manufacturing and the anlysis of this data to make automatic decisionmaking. Decicion that could have an automatic effect. 
    \section{IoT of Industry 4.0}
    The idea is that the more data is accumilated the better decisions could be made. The collection of data is partly done via IoT (Internet of Things).
    
    \section{No more paper}
    \section{Realtime metrics}
    \section{Big data analytics}
    
    \section{Industrial IoT}
    The winner in manufacturing if most effective at allocating resources, producing produce and distributing. Behind all this is human activity that has been automated over time. From the first revolution with machinery automating producing parts, to second electrifing revolution giving benefits to all parts. Then digital age further streamlined the parts nessaary in manufacturing. All it's about is information to make action. THe faster the allocation of information and the computing act to make action can happen the more effective the manufacturing. Then ofc the computing needs to be correct. For example:
    \\Give an example where information given about a demand can be found by manufacturer which push production and dsitribution to correct region\\
    
    \section{Conclusion}

\printbibliography
\end{document}