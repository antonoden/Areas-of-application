
\documentclass[article,a4paper]{IEEEtran}
\usepackage{lipsum}
\usepackage[backend=biber]{biblatex}

\addbibresource{refs.bib}
\title{Internet of things areas of application}
\author{
\IEEEauthorblockN{Anton Odén}\\
\IEEEauthorblockA{Dept. of Maths and Computer Science\\Karlstad University\\
651 88 KARLSTAD, Sweden}\\
anton.oden@outlook.com
}

\begin{document}

\maketitle
\begin{abstract}
    \end{abstract}
    
    \section{Introduction}
    
    \section{Industrial IoT}
    The winner in manufacturing if most effective at allocating resources, producing produce and distributing. Behind all this is human activity that has been automated over time. From the first revolution with machinery automating producing parts, to second electrifing revolution giving benefits to all parts. Then digital age further streamlined the parts nessaary in manufacturing. All it's about is information to make action. THe faster the allocation of information and the computing act to make action can happen the more effective the manufacturing. Then ofc the computing needs to be correct. For example:
    \\Give an example where information given about a demand can be found by manufacturer which push production and dsitribution to correct region\\

    \section{Transportation}

    \section{Smart Grid}
    The electrical grid is full of sensors to be able to maintain it stable. 
    \section{Smart Homes}
    A big part of energy consumptions of electrical power in our homes is due to heating. The heating part of a home could take a lot of different forms. In countryside Sweden taking heat from the ground parallel with air is comon. But without smartness in the system the heating isn't adapting to our need of heat. Many hours of the day we could be away from home. These hours often follow a pattern. Working nine to five or weekendly trips it is unnessasary to heat up our homes to human comfortable level. Also highly personal, but night temperature could also be lowered when sleeping. Could our home sense these time of day and adapt to our presence we have added some smartness unto the heatingsystems. 
    
    Together with the smart grid giving energy prices based on supply and demand our smart home could contribute by changing consumption of devices to cheap energy price times. For example could the dish- and washing machine be activated or scheduled to activate during peak cheap hours. Often during nights. Same with charging of our electrical cars. Which has been exampled as a power battery that could be very beneficial for our smart grid. 
    \section{Smart Cities}
    Sensing how crowded traffic is in parking lots other cars a redirected to less crowded spaces. This is already active today. 
    \section{Healthcare}
    
    \section{Agriculture}
    
    \section{Conclusion}

\printbibliography
\end{document}